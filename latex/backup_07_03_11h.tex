\documentclass[journal]{IEEEtran}
\usepackage{amsmath,amsfonts}
\usepackage{algorithmic}
\usepackage{algorithm}
\usepackage{array}
\usepackage[caption=false,font=normalsize,labelfont=sf,textfont=sf]{subfig}
\usepackage{textcomp}
\usepackage{stfloats}
\usepackage{url}
\usepackage{verbatim}
\usepackage{graphicx}
\usepackage{cite}

\begin{document}

\title{Transfer learning between Sentinel-1 acquisition modes enhances \\ the segmentation of natural oil slick events in the Arctic}

\author{Julien Vadnais,~\IEEEmembership{Graduate Student Member,~IEEE, }Benjamin Aubrey Robson, Christian Haug Eide, Rune Mattingsdal, Malin Johansson
        % <-this % stops a space
\thanks{Manuscript received April 19, 2021; revised August 16, 2021. This research was carried out in the context of the InFluSe project (NFR Project No.?).}

%grouped affiliations
% \thanks{Julien Vadnais, Benjamin Aubrey Robson, and Christian Haug Eide are with the Department of Earth Science, University of Bergen, 5020 Bergen, Norway.
% Rune Mattingsdal is with the Norwegian Offshore Directorate, 9406 Harstad, Norway.
% Malin Johansson is with the Department of Physics and Technology, The Arctic University of Norway, 9037 Tromsø, Norway. Corresponding author email address: julien.vadnais@uib.no} 

%separate affilations
\thanks{Julien Vadnais, Benjamin Aubrey Robson, and Christian Haug Eide are with the Department of Earth Science, University of Bergen, 5020 Bergen, Norway.}
\thanks{Rune Mattingsdal is with the Norwegian Offshore Directorate, 9406 Harstad, Norway.}
\thanks{Malin Johansson is with the Department of Physics and Technology, The Arctic University of Norway, 9037 Tromsø, Norway.}
\thanks{Corresponding author email address: julien.vadnais@uib.no. This article has supplementary available in the GitHub repository at $https:/github.com/julvad/temp_repo_a0$}

\thanks{}% <-this % stops a space
}

% The paper headers
\markboth{\LaTeX\ PDF, IEEE GEOSCIENCE AND REMOTE SENSING LETTERS,~Vol.~14, No.~8, March~2025}%
{Shell \MakeLowercase{\textit{et al.}}: A Sample Article Using IEEEtran.cls for IEEE Journals}

\IEEEpubid{\copyright This work is licensed under CC BY 4.0.}
% Remember, if you use this you must call \IEEEpubidadjcol in the second
% column for its text to clear the IEEEpubid mark.

\maketitle

\begin{abstract}
    This document describes the most common article elements and how to use the IEEEtran class with \LaTeX \ to produce files that are suitable for submission to the IEEE.  
    IEEEtran can produce conference, journal, and technical note (correspondence) papers with a suitable choice of class options. 
    This document describes the most common article elements and how to use the IEEEtran class with \LaTeX \ to produce files that are suitable for submission to the IEEE.  
    IEEEtran can produce conference, journal, and technical note (correspondence) papers with a suitable choice of class options. 
    This document describes the most common article elements and how to use the IEEEtran class with \LaTeX \ to produce files that are suitable for submission to the IEEE.  
    IEEEtran can produce conference, journal, and technical note (correspondence) papers with a suitable choice of class options. 
    This document describes the most common article elements and how to use the IEEEtran class with \LaTeX \ to produce files that are suitable for submission to the IEEE.  
    The entirety of the data and Python code used in this study are provided in an open-access online repository.
\end{abstract}

\begin{IEEEkeywords}
    Article submission, IEEE, IEEEtran, journal, \LaTeX, paper, template, typesetting.
\end{IEEEkeywords}

\section{Introduction}
\IEEEPARstart{M}{onitoring} offshore oil bodies has long  been one of the principle applications of remote sensing oceanography. 
The first observations of oil slicks in satellite imagery go back to the early 1970s \cite{deutschUseLandsatData1977,stumpfERTS1ViewsOil1974}, 
making an important shift in the topic which previously relied on airborne remote sensing systems. Early research, and most of the attention on the topic until now, 
has been dedicated to anthropogenic oil spills, but oil slicks can also originate from natural processes. 
Natural oil slicks hint at an underlying seep on the seabed beneath, resulting from the upward migration and the release of hydrocarbons from reservoir rocks to the surface
\cite{huntPetroleumGeochemistryGeology1979,linkSignificanceOilGas1952,juddSeabedFluidFlow2009,marshakEarthPortraitPlanet2005}.

Natural seepage is a significant contributor of oil entering marine environments \cite{kvenvoldenNaturalSeepageCrude2003}, yet the data required to update quantitative
estimates of its global scale remain insufficient, especially outside of the United States continental shelf \cite{nationalacademiesofsciencesengineeringandmedicineOilSeaIV2022}. Data scarcity, 
due to difficulties in monitoring remote offshore areas and the challenge of continuous temporal observation, has limited possibilities to study the ecosystem contributions 
and global behavior of natural hydrocarbon seepages. Among satellite-mounted imaging sensors, synthetic aperture radar (SAR), is well suited to this task due to its capacity to operate in any lighting conditions, 
its fine spatial resolution and its wide swath coverage. Our capacity for detecting a slick in a given SAR image relies on the damping ratio, the contrast between radar backscatter in slick and oil-free pixels 
\cite{hovlandSlickDetectionSAR1994,quigleyInvestigationDampingRatio2023}. By reducing sea surface roughness, which reduces radar backscatter, bodies of oil can be discernible from surrounding waters \cite{brekkeSAROilSpill2020,fingasReviewOilSpill2018,alpersOilsSurfactants2004}. 
Numerous frameworks have been proposed to leverage SAR for automatizing slick detection using computer vision algorithms.
They face a common challenge which has been an ongoing topic of research over the past three decades: the presence of many oil slicks lookalikes including ocean eddies, biogenic slicks and newly-formed ice
\cite{johanssonCanMineralOil2020,hovlandSlickDetectionSAR1994,alpersOilsSurfactants2004,alpersOilSpillDetection2017,espedalSatelliteDetectionNatural1996}. When exposed to a large number of labeled examples, 
deep learning algorithms, currently the benchmark in computer vision, can capture complex shape- and texture-related features for image classification or segmentation tasks \cite{goodfellowDeepLearning2016}. 
Although this approach can prove a powerful discriminator of slick lookalikes, it is known that when trained with limited sample sizes, models struggle to extract more abstract, more generalizable 
higher-level features \cite{bengioDeepLearnersBenefit2011,bengioDeepLearningRepresentations2012}. 

The issue of limited labeled data is inherent to machine-learning-based SAR oil slick mapping. 
\IEEEpubidadjcol 
Firstly, depending on thickness and weather conditions, slicks will only last a few hours on the surface, on average 3 
\cite{jatiaultMonitoringNaturalOil2017} to 6 hours \cite{daneshgaraslHindcastModelingOil2017,oreillyDistributionMagnitudeVariability2022}, shorter than what was previously thought 
\cite{macdonaldNaturalOilSpills1998,macdonaldNaturalUnnaturalOil2015}. Furthermore, the visibility of slicks is dependent on surface wind conditions. Outside of a restrained wind-speed window, 
the sea either exhibits too many or too few capillary waves to produce a discriminative damping ratio \cite{quigleyInvestigationDampingRatio2023,sausDetectionDelineationProduced2021,gadeImagingBiogenicAnthropogenic1998}. 
Lastly, slick events are highly intermittent. We observed considerable variation in seepage rates across both monthly and yearly timescales, with most active slicks having an occurence rate below 10\% 
\cite{jatiaultNaturalOilSeep2024,oreillyDistributionMagnitudeVariability2022}. Some seeps might stay dormant for several months or years before leaking oil again. In two different regions, it was found that only
half of the seeps show activity \cite{jatiaultMonitoringNaturalOil2017,garcia-pinedaRemotesensingEvaluationGeophysical2010}, and it is not unsound to believe that this value might be even higher, 
given that such seeps are logically less likely to be discovered.

In this paper, we assess the potential of transfer learning for enhancing the automatic mapping of natural slicks. 
We use deep learning models pre-trained on Sentinel-1 SAR observations in the North Sea, and then fine-tune models trained and validated on two known slick events in the Arctic, on which we have only limited observations.
Over the Arctic, Sentinel-1 images are mostly acquired in the extra-wide (EW) mode with a dual HH/HV polarization. This differs from most previous studies on SAR oil slick mapping, which more often used the 
interferometric-wide (IW) mode with a dual VV/VH polarization. Little work has until now addressed transfer learning between SAR acquisition modes, and natural oil slick events represent 
a prime example for a real-life application of deep learning segmentation with restricted labels. 

\section{Data and methods}
\subsection{Data}
Evidence of two intermittent seep areas were recently documented in the Norwegian territorial waters. The first is located west of the Svalbard archipelago, offshore Prins Karls Forland (10.42°E; 78.49°N). 
It was revealed and thoroughly studied by Panieri et al. \cite{panieriArcticNaturalOil2024}. Oil slicks extending from a single seeping point were manually delineated in 26 images in 2020 and 2021. 
The second area, located in the Barents Sea, was documented in \cite{serovWidespreadNaturalMethane2023} (but see also \cite{ivanovSearchDetectionNatural2020}). Natural oil slicks from 90 images between 2015 and 2020 
were mapped over a spread-out area (31.33 - 32.37°E, 75.15°N - 75.35°N) in Sentralbanken high, where is located an active petroleum system with prospective hydrocarbon potential, though it remains largely underexplored."
\cite{lundschienNorthBarentsComposite2025}. REPHRASE Sentinel-1 extrawide images using the horizontal co-polarization (HH) band. This is our \textit{target domain}.


The North Sea to hydrocarbon seepages \cite{juddSeabedFluidFlow2009,hovlandSeabedPockmarksSeepages1988}

In this, work, labeled slicks in S1 IW VV rasters constitute the \textit{source Domain},















\section{Conclusion}
The conclusion goes here.


\section*{Acknowledgments}
Julien Vadnais received financial support from the Doctoral Research Scholarship of the Fonds de recherche du Québec.

{\appendix[Appendix A]
Write appendix or enter here
}

% \section{References}
\bibliographystyle{IEEEtran}
\bibliography{references}


\end{document}